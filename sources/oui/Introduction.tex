\chapter{Introduction}


La sécurité informatique au sein d’une entreprise est aujourd'hui devenue le domaine avec le plus grand enjeu. Il faut donc du personnel spécialisé dans ce dernier afin de la mettre en place. On se rend facilement compte que le meilleur moyen de s’améliorer dans ce milieu est dans un premier temps de se documenter puis de réaliser des attaques pour savoir par la suite comment s'en défendre. C’est à ce moment-là que le "Capture The Flag" ou bien "Capturer Le Drapeau" intervient. A l’origine, un CTF est un jeu à l’air libre où deux équipes s’affrontent pour s’emparer du drapeau de l’adversaire. On peut alors s’apercevoir que le monde informatique est semblable à celui réel. Un CTF est alors un concept ayant pour but d’infiltrer une machine cible et de trouver un document, le "flag". Le CTF s’est démocratisé en 1996 lors des premières compétitions organisées par la DEF CON. La DEF CON est la convention de hackeur la plus connue du monde. \\
Les CTF s’inspirent de la vraie vie même si cela reste un terrain d'entraînement. Les CTF reposent sur plusieurs domaines qui sont : le reverse engineering, l’exploitation web, le forensic, le réseau, la cryptographie, la sécurité mobile, la stéganographie et d'autres encore.
Tous ces domaines sont les piliers de la sécurité informatique. Il faudra donc être polyvalent afin d’exploiter les failles et de résoudre un CTF. Nous allons donc voir lors de ce cours les différents moyens de parvenir à nos fins.\\
Nous tenons à rappeler qu'il est strictement interdit de pratiquer de l'ethical hacking sur un réseau qui n'est pas le vôtre et où vous n'avez pas l'autorisation de réaliser une attaque. Ce cours a été créé dans un but éducatif et non malveillant.